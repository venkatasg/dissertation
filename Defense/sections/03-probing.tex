\begin{frame}[c]\frametitle{Back to the LIB}
    \begin{table}[t]
    \centering
    \begin{tabular}{lll}
     \toprule
        \textbf{} & \textbf{In-group} & \textbf{Out-group} \\ \midrule
        socially desirable & abstract & concrete \\ \midrule
        socially undesirable & concrete & abstract \\ 
    \bottomrule
    \end{tabular}
    \caption{Predicted language variation in the LIB.}
    \label{tab:lib}
\end{table}

    
    \pause\vfill
    
    But LIB defines abstractness ad-hoc based on word-lists of predicates --- all adjectives are more abstract than all verbs, etc. Social desirability is a vague notion as well. 
    
    \vfill
    
    Can we do better?

\end{frame}


\begin{frame}[c]\frametitle{Intergroup bias broadly}

   With specificity and affect as holistic measures, we can design a new hypothesis quadrant:
    
    \vfill
    \begin{table}[t]
    \centering
    \begin{tabular}{lll}
     \toprule
        \textbf{} & \textbf{in-group} & \textbf{out-group} \\ \midrule
        positive affect & low specificity &  high specificity \\ \midrule
        negative affect & high specificity & low specificity \\ 
    \bottomrule
    \end{tabular}
    \caption{Predicted language variation in our more general formulation, using specificity and affect}
    \label{tab:ib}
\end{table}

    
    \vfill\pause
    
    We test this narrow hypothesis in \textcolor{AmmaBlue}{\cite{govindarajan-etal-2023-counterfactual}}, by \textbf{probing what our model learned}.
\end{frame}


% \begin{frame}[c]\frametitle{Probing Hypothesis}
% 
%     \begin{figure}[t]
%         \begin{tikzpicture}[node distance = 0.1\textwidth, auto]
    % Place nodes
    \node (middle-plot) {
        \begin{tikzpicture}
            \begin{axis}[ybar, ymin=0,ymax=100,
                         symbolic x coords={+ve,-ve},xtick=data,  /pgf/bar width=2em, width=0.25\textwidth,height=0.2\textheight, tick label style={font=\small}, yticklabels=None, enlarge x limits=0.35, ytick style={draw=none}, xlabel={\small Affect}, ylabel={\small\% in-group}]
                \addplot+[ybar, color=TolIndigo!50]
                    coordinates {(+ve,40) (-ve,60)};
            \end{axis}
        \end{tikzpicture}
    };
    \node [left of=middle-plot, xshift=-0.2\textwidth] (left-plot) {
        \begin{tikzpicture}
            \begin{axis}[ybar, ymin=0,ymax=100, ylabel={Hate Speech \%},
                         symbolic x coords={+ve,-ve},xtick=data,  /pgf/bar width=2em, width=0.25\textwidth,height=0.2\textheight, tick label style={font=\small},yticklabels=None, enlarge x limits=0.35, ytick style={draw=none}, xlabel={\small Affect}, ylabel={\small\% in-group}]
                \addplot+[ybar, color=TolIndigo!50]
                    coordinates {(+ve,20) (-ve,90)};
            \end{axis}
        \end{tikzpicture}
    };
    \node [right of=middle-plot, xshift=0.2\textwidth] (right-plot) {
        \begin{tikzpicture}
            \begin{axis}[ybar, ymin=0,ymax=100,
                         symbolic x coords={+ve,-ve},xtick=data, /pgf/bar width=2em, width=0.25\textwidth,height=0.2\textheight, tick label style={font=\small},yticklabels=None, enlarge x limits=0.35, ytick style={draw=none}, xlabel={\small Affect}, ylabel={\small\% in-group}]
                \addplot+[ybar, color=TolIndigo!50]
                    coordinates {(+ve,90) (-ve,10)};
            \end{axis}
        \end{tikzpicture}
    };
    \node [block, below of=middle-plot, yshift=-0.1\textheight] (bert-1) {\small $h_i$};
    \node [block, left of=bert-1, xshift=-0.2\textwidth] (bert-pos) {\small $h_i^s$};
    \node [block, right of=bert-1, xshift=0.2\textwidth] (bert-neg) {\small $h_i^g$};
    
    \node [block, below of=bert-1, yshift=-0.1\textheight] (bert) {\small  $h_i$};
    \node [cloud, below of=bert-pos, yshift=-0.1\textheight] (pos) {\shortstack{\small  push token embeddings\\\small  towards high specificity}};
    \node [cloud, below of=bert-neg, yshift=-0.1\textheight] (neg) {\shortstack{\small  push token embeddings\\\small  towards low specificity}};
    
    \node [tweet, below of=bert, yshift=-0.05\textheight] (tweet) {\small Thanks, \@USER for joining me in demanding \#PaperBallotsNOW};
    
    % Draw edges
    \path [line] (tweet) -- node {\small  get embeddings with BERTweet}(bert);
    \path [line] (bert) -- (bert-1);
    \path [line] (pos) -- (bert-pos);
    \path [line] (neg) -- (bert-neg);
    \path [line] (bert) -- (pos);
    \path [line] (bert) -- (neg);
    \path [line] (bert-1) -- node[pos=0.5,left] {\small In or Out-Group?}(middle-plot);
    \path [line] (bert-neg) -- node[pos=0.5,left] {\small In or Out-Group?}(right-plot);
    \path [line] (bert-pos) -- node[pos=0.5,left] {\small In or Out-Group?}(left-plot);
    \draw[latex'-latex'] (left-plot) -- node {\tiny Compare}(middle-plot);
    \draw[latex'-latex'] (middle-plot) -- node {\tiny Compare}(right-plot);
\end{tikzpicture}

%         \label{fig:flowchart-probing}
%     \end{figure}
% 
% \end{frame}

\begin{frame}[c]\frametitle{Specificity Results}

    \begin{figure}
        \centering
        \begin{tikzpicture}
	\begin{axis}[
		xlabel={INLP iterations},
		% ylabel={F1 Score/\In-group},
        axis x line*=bottom,
        y axis line style= { draw opacity=0 },
        ymajorgrids=true,
        xtick={8,16,24,32,40,48,56,64},
        enlarge x limits=-1,
        ymin=0,
        ymax=100,
        height=0.4\textheight,
        width=\linewidth,
        legend columns=-1,
        legend style={font=\small, at={(0.5,1.2)},anchor=north, draw=none, /tikz/every even column/.append style={column sep=0.03\columnwidth}},
          legend entries={Push high specificity,
                          Push low specificity,
                          Controls
                          }
    ]
    \addplot[only marks, thick, color=TolTeal, mark=*, mark size=2pt] table [x=inlp, y=posspec, col sep=tab] {data/test.tsv};
    
    \addplot[only marks, thick, color=TolWine, mark=triangle*, mark size=2pt] table [x=inlp, y=negspec, col sep=tab] {data/test.tsv};
    
    \addplot[only marks, thick, color=black!60, mark=x, mark size=2pt] table [x=inlp, y=posspec-control, col sep=tab] {data/test.tsv};
    
    \addplot[only marks, thick, color=black!60, mark=x, mark size=2pt] table [x=inlp, y=negspec-control, col sep=tab] {data/test.tsv};
	\end{axis}
\end{tikzpicture}

        \label{fig:specificity}
    \end{figure}

\end{frame}

\begin{frame}[c]\frametitle{Takeaways}

\begin{itemize}
    \itemsep=\baselineskip
    \item Our narrow, novel intergroup hypothesis didn't replicate in the data, but that's ok! It brings me back to the bigger picture.\pause
    \item We need more natural language data to \textbf{discover} linguistic variations in how the intergroup bias is expressed.\pause
    \item We need to account for the influence of real-world events which is the source of affect/emotion.
\end{itemize}


\end{frame}
