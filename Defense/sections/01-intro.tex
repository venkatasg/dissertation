\begin{frame}[plain, standout]\frametitle{}
    How is \textbf{in-group} speech different from \textbf{out-group} speech?
\end{frame}

\begin{frame}[c]\frametitle{Framing Bias}

    Most work in NLP approaches bias as \alert{negative or perjorative language use towards an individual or group} based on demographics.\pause

    \vfill

    However, research in psychology and social science suggests that bias is difference in behavior situated in relationships between people, and context. \textbf{All language use is biased}.

    \vfill

    How do we bring this insight into our work?

    \blfootnote{\cite{van2009society, beaver2018toward, kaneko-bollegala-2019-gender, sap-etal-2020-social, webson-etal-2020-undocumented}}

\end{frame}


\begin{frame}[c]\frametitle{Linguistic Intergroup Bias}

    The LIB hypothesis tries to explain the persistence of stereotypes through systematic language variation between \textbf{in-group} and \textbf{out-group} language.
    
    \pause
    \vfill
    
    LIB hypothesizes that abstract predicates are used when a description \textbf{conforms to stereotype}.
    
    \ex. \a. The man police want to talk to probably \textbf{\alert{hit}} the victims.
         \b. The man police want to talk to probably \textbf{\alert{hurt}} the victims.
         \b. The man police want to talk to probably \textbf{\alert{hated}} the victims.
         \b. The man police want to talk to is probably \textbf{\alert{violent}}.

    \blfootnote{\cite{maass_linguistic_1999, gorham_news_2006}}
\end{frame}


\begin{frame}[c]\frametitle{Intergroup Bias}
        \centering\large
        We can study systematic differences in interpersonal language \emph{inspired by the LIB}, and this can be an \textbf{effective framing} of social bias --- intergroup bias.
\end{frame}

\begin{frame}[c]\frametitle{Main findings}

    \begin{enumerate}
        \itemsep=\baselineskip
        \item Intergroup bias can be analyzed through decomposition into \textbf{relationship} (in-group vs. out-group) and \textbf{emotion} in political tweets.\pause
        \item By grounding intergroup bias in a robust description of events preceding an utterance, we find that \textbf{form of referent varies linearly} with the grounded descriptions.
    \end{enumerate}

\end{frame}


\begin{frame}[c]\frametitle{Outline of talk}

\begin{enumerate}
    \itemsep=\baselineskip
    \item Intergroup bias in political tweets.\pause
    \item Counterfactual probing for intergroup bias.\pause
    \item Grounding intergroup bias in football comments.
\end{enumerate}

\end{frame}
