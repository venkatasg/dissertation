Social bias in language is generally studied by identifying undesirable language use towards a specific demographic group, but we can enrich our understanding of communication by re-framing bias as \emph{differences in behavior situated in social relationships} --- specifically, the intergroup relationship between the speaker and target reference of an utterance. This dissertation aims to understand the nature of systematic variation between in-group and out-group speech --- the intergroup bias. Chapter~\ref{chapter:twitter} describes systematic interactions between intergroup bias and interpersonal emotion, and finds that learned neural models can learn to classify the intergroup relationship in tweets with information about the speaker and target masked, out-performing expert annotators. To decipher what human-interpretable linguistic features are learned by these models, Chapter~\ref{chapter:probing} describes probing experiments to understand if two pragmatic features --- affect and specificity --- are \emph{used} by these models in differentiating in-group from out-group utterances. While affect and specificity have directional impacts on model prediction that align with our intuitions (positive affect utterances are more likely to be in-group), we found no interaction between these two variables, as we hypothesized generalizing from the Linguistic Intergroup Bias. Experiments up until this point also demonstrated the need for interpersonal language use \emph{grounded} in non-linguistic world-state. To address this, Chapter~\ref{chapter:football} investigates online comments from live game threads on forums (subreddits) dedicated to individual NFL teams. We find systematic linear relationships between the \emph{form of referent} used to describe in-group and out-group teams, and the win probability for the in-group (game state) at the time of utterance. The better the state-of-the-world for the in-group, fans refer to the in-group less and the out-group more. Fans are also more likely to abstract away from events of the game and not refer to the in-group or out-group at all, choosing instead to express excitement, the more likely the in-group is to win. State-of-the-world as expressed in the win probability for the in-group turns out to be a well calibrated metric for intergroup bias, as evidenced by the linear relationships between several referent forms and win probability for the in-group. Data-driven analysis and experimentation with modern NLP tools in this dissertation thus revealed a form of intergroup bias ignored in previous literature, and outlines a promising picture of utilizing statistical information processing paired with careful data curation and analysis to understand subtle changes in human social behavior
