Social bias in language is generally studied by identifying undesirable language use towards a specific demographic group, but we can enrich our understanding of communication by re-framing bias as \emph{differences in behavior situated in social relationships} --- specifically, the intergroup relationship between the speaker and target reference of an utterance. This dissertation aims to understand the nature of systematic variation between in-group and out-group speech --- the intergroup bias. Chapter~\ref{chapter:twitter} describes systematic interactions between intergroup bias and interpersonal emotion, and finds that learned neural models can learn to classify the intergroup relationship in tweets with information about the speaker and target masked, out-performing expert annotators. While probing experiments in Chapter~\ref{chapter:probing} failed to identify utterance-level features that explain the nature of this bias, it revealed the need for interpersonal language use \emph{grounded} in non-linguistic world-state. Chapter~\ref{chapter:football} investigates online comments on forums dedicated to specific NFL teams parallel with live game data --- we find systematic \emph{linear} relationships between the form of referent used to describe in-group and out-group teams, and the game-state at the time of utterance. Overall, the studies in this dissertation detail how modern NLP techniques, in-concert with careful data interpretation, can aid in the discovery of subtle and systematic variations in communicative behavior.
