\chapter{Introducing intergroup bias}
\label{chapter:intro}

Currently, most work studying bias in NLP situates bias as negative or pejorative language use towards an individual or group based on traits like race, gender, etc~\citep{kaneko-bollegala-2019-gender, sheng-etal-2019-woman, sap-etal-2020-social, webson-etal-2020-undocumented, Pryzant_DiehlMartinez_Dass_Kurohashi_Jurafsky_Yang_2020, sheng-etal-2020-towards}.  While these approaches greatly advance our understanding of bias in language and its impact and mitigation in NLP, focusing on specific demographic dimensions or an individual's intent is limiting and not always practical. Research in psychology and social science suggests a different perspective. Bias can be seen as a relationship between people and groups, situated in context~\citep{van2009society}; as such, bias refers to differences in behavior (in this case language use) as a result of differences in the relationship between speaker and target. The language we produce is biased in one way or another, whether we intend to or not, and whether that bias is positive, negative, or not clearly associated with any valuation~\citep{beaver2018toward}.

So how do we study bias from this perspective? By accepting that the language we produce is biased, and is dependent on our social identity and relationships to the people we talk about (and to), we can focus on various aspects of modeling the subtle influence of social identity relationships on our linguistic production, as well as how language shapes and informs social identity in turn~\citep{eckert2012three}. One such form of social meaning~\citep{Hall-Lew_Moore_Podesva_2021,beltrama2020social}, that will be the focus of this dissertation is \textbf{intergroup bias}. The next section introduces the psycho-linguistic literature behind studies of the Linguistic Intergroup Bias, and its deficiencies that makes drawing inferences on social communication more broadly hard. By focusing on data-driven studies of in-group versus out-group speech in the wild, this dissertation aspires holistic understanding of how intergroup social differences has a bearing on the form, and meaning, of the language we produce.

\section{The Linguistic Intergroup Bias}
\label{sec:intro-lib}

The Linguistic Intergroup Bias hypothesis tries to explain why stereotypes persist, and how they are transmitted sub-consciously in daily conversation. In an intergroup context, and focusing on actions that are considered stereotypical for a group, socially desirable in-group behaviors and undesirable out-group behaviors are encoded at a higher level of \textbf{abstraction}, whereas socially undesirable in-group behaviors and desirable in-group behaviors are encoded at a lower level of abstraction (described in Table~\ref{tab:lib}. A crucial underpinning of the theory is the Linguistic Category Model~\citep{semin_cognitive_1988}.

\begin{table}[t]
    \centering
    \begin{tabular}{lll}
     \toprule
        \textbf{} & \textbf{In-group} & \textbf{Out-group} \\ \midrule
        socially desirable & abstract & concrete \\ \midrule
        socially undesirable & concrete & abstract \\ 
    \bottomrule
    \end{tabular}
    \caption{Predicted language variation in the LIB.}
    \label{tab:lib}
\end{table}


The Linguistic Category Model classifies predicates (words that can be used to describe people; like adjectives and verbs) on a scale of increasing abstraction -- from verbs that are most concrete, to verbs that are less concrete, to adjectives that are most abstract. Abstract words (and thus statements) are taken to imply greater {\scshape temporal stability} and revealing of the character of a referent than their concrete counterparts. This provides an explanation for the persistence of bias and stereotypes through LIB -- people tend to use abstract statements to talk about desirable in-group and undesirable out-group behaviors since they are \textbf{potentially more informative of the referent} (that is, indicative of future behavior). Consider the following utterances regarding a subject \textit{Johnson} (examples taken from \citet{gorham_news_2006}):

\ex. \label{ex:exs} \a. The man police want to talk to probably hit the victims.\label{ex:dav}
     \b. The man police want to talk to probably hurt the victims.\label{ex:iav}
     \b. The man police want to talk to probably hated the victims.\label{ex:sv}
     \b. The man police want to talk to is probably violent .\label{ex:adj}
     
\textit{hurt} in \ref{ex:dav} is a direct action verb; \textit{hurt} in \ref{ex:iav} is an interpretive action verb; \textit{hated} in \ref{ex:sv} is a stative and; \textit{violent} is an adjective. Moving from \ref{ex:dav} to \ref{ex:adj}, one can see how the information about the subject increases, while the information regarding a specific situation \textit{decreases}. Thus, the \emph{abstractness} of predicates increases from \ref{ex:dav} to \ref{ex:adj} according to the LCM.

The LIB uses the LCM ladder of abstraction to predict linguistic behavior: a speaker is more likely to describe an \emph{out-group} individual with abstract predicates if the actions of the individual are socially undesirable, or negative stereotype congruent. Thus, white participants in the study from ~\citet{gorham_news_2006} were more likely to describe the person whose picture they saw in a news report (whose race was varied as the experimental condition) using \ref{ex:sv} or \ref{ex:adj} if they were black (thus making them out-group), since it reinforces their negative stereotypes of African Americans. The converse holds for in-group referential utterances --- white participants are more likely to use \ref{ex:dav} or \ref{ex:iav} to describe the person from the news report if he is presented as white.

There has been a wealth of work in psychology and psycholinguistics reproducing LIB in various domains such as crime reports and racial bias~\citep{gorham_news_2006}; political news and party bias~\citep{anolli_linguistic_2006}; as well as work exploring how LIB interacts with a speaker's prejudical attitudes~\citep{schnake_modern_1998, greenwald_implicit_2006}. The LIB's strengths lie in the simplicity of its predictions, succinctly described in Table~\ref{tab:lib}, and its focus on \emph{abstraction} as a language feature --- offering an attractive tie-in to cognitive mechanisms underlying prejudice and stereotypes. However, its weaknesses, closely tied to its strengths, also prevent it from being used to draw inferences of social communicative behavior at scale.

The LCM, upon which the LIB rests, while useful as an analysis of linguistic abstraction, suffers from a few drawbacks. The distinction between some of the classes, say DAV and IAV, are not very linguistically motivated. DAVs are said to refer to `objective descriptions of observable behaviors', all usages of that verb sharing a \textit{physically invariant} component, while IAVs are said to refer to a general class of behaviors with positive or negative connotations. It remains to be investigated whether these definitions refer to something real (are DAVs less polysemous than IAVs?), but it would appear that a scale of abstractness would be more suited to this task. Some DAVs are more connotative than others (\textit{hit} in \ref{ex:dav} versus \textit{perform}), whereas even within adjectives, some (like \textit{athletic}) are more concrete than others (\textit{loyal}).

Furthermore, the LCM constructs abstraction as simply a function of the verb/predicate. This is inherently limiting --- can the subtle intergroup biases not be reflected in other parts of the utterance, or in the utterance as a whole? The simplicity of the LIB formulation is compounded by the ad-hoc nature of the defined axes of variation, especially the social desirability angle. 

The LIB is a useful framework for analysis of utterances under very specific conditions --- a focus on eliciting utterances from participants in constrained experiments, hand-coding social desirability as well as abstractness of predicates, and a focus on attitudes that are considered stereotypical of groups at the time. Real-world utterances about, or directed at, other people/groups show much more variation and diversity --- does intergroup bias systematically influence real-world language use? This thesis is my program towards answering this question; To \textbf{characterize intergroup bias in real-world utterances through data curation, analysis and computational modeling}.

\section{Outline}

This dissertation concerns intergroup bias in online conversation, which I aim to understand and study through large-scale data analysis and modeling. Chapter~\ref{chapter:twitter} introduces the notion of intergroup bias, and motivates why we need to study it in addition to demographically-defined social biases. It defines various terms and concepts, and describes our first dataset of focus --- tweets by US Congress members directed at other members. We find intrinsic statistical relationships between emotion and intergroup bias, which can further be learned and recognized by models for identifying if tweets are directed in-group or out-group with no knowledge of entities involved. This chapter was published as a paper at EACL 2023~\citep{govindarajan-etal-2023-people}. 

Chapter~\ref{chapter:probing} builds on top of the dataset and investigation in Chapter~\ref{chapter:intro}, and examines what \emph{systematic}, \emph{linguistic} changes can explain the differences between in-group and out-group speech. Extrapolating from the LIB, we define a new quadrant of intergroup language variation of the intergroup bias, operationalizing the ad-hoc axes of LIB towards automatically inferable, linguistically grounded variables. Through probing experiments, we discovered the limitations of a hypothesis driven approach towards \emph{discovering} unknown, subtle intergroup variations in real-world language use, as well as the need for \textbf{grounding} our utterances in descriptions of events precipitating/preceding the utterance. While the hypothesis driven approach in this paper (published in the Findings of ACL 2024~\citep{govindarajan-etal-2023-counterfactual}) failed, it was instructive and steered our focus to the issue of grounding and reference form itself.

Chapter~\ref{chapter:football} addresses this by introducing a new dataset of interpersonal utterances --- over 6 million comments by NFL fans on live-game threads, grounding these interpersonal comments in events. Building upon the rich literature from the NFL statistics community, we utilize a real-valued number (the win probability) that succinctly describes the events preceding an utterance as it pertains most towards the intergroup bias --- how well are things going for my in-group? We also introduce a novel way of modeling the intergroup bias, by tagging words in an utterance that refer to relevant entities as in-group or out-group. 

Analyzing over 200,000 comments tagged with intergroup tags using the statistical information processing capabilities of modern Large Language Models (LLMs) revealed a hidden variation not captured by the LIB: the \textbf{form of referent} when talking about the in-group or out-group changes systematically over time. Fans are more likely to abstract away from referencing a specific individual or team, towards a description of events in general, the more likely their team (the in-group) is to winning. Furthermore, this trend is remarkably linear over win probabilities. References to the out-group remain steady across win probabilities.

Overall, the findings in this thesis constitute the first data-driven large scale study of intergroup bias in real-world language use. The findings add much needed color and linguistic rigor to the LIB hypothesis. Future work needs to expand this work to more domains, to gain a holistic understanding of how social structures and relationships mediate our language use subconsciously.
