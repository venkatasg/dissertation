\chapter{Prompts and instructions for few-shot learning}
\label{appendix:prompts}

For instruction fine-tuning of Flan-T5, the following instructional prefix detailing the tagging process was fed to each datapoint to be tagged:

\begin{verbatim}
Tag references to entities as in-group ([IN]), out-group 
([OUT]) or other ([OTHER]) in live, online sports comments 
during NFL games. The input is the comment, the in-group 
team the commenter supports, the out-group opponent team, 
and the win probability for the in-group at the time of 
the comment. The win probability is the probability of 
the in-group winning the game at the time of the comment 
- if the win probability is high, the in-group team is 
probably doing well and going to win. Using knowledge of 
American football and contextual language understanding, 
identify words and phrases denoting entities (players, 
teams, city names, sub-groups within the team) that 
refer to the in-group ([IN] - team the commenter supports),
out-group ([OUT] - the opponent) or other teams ([OTHER] 
- some other team in the NFL that is not the in-group or 
the opponent), with respect to the commenter. Return the 
TARGET comment itself with relevant words/phrases replaced 
with the respective tags ([IN], [OUT] or [OTHER]), the 
list of words/phrases that are to be tagged 
(REF_EXPRESSIONS), and an EXPLANATION justifying the 
choice of REF_EXPRESSIONS in your final output.

Each sentence in a comment is separated by a [SENT] token. 
Sometimes a sentence in the comment will be about the 
in/out/other group but not have an explicit word/phrase 
that refers to the group; In such cases, tag the [SENT] 
token for that sentence with the corresponding tag label.

Here are 6 examples, with EXPLANATION being a reasonable 
reason for why TARGET is the correct tagged output for 
COMMENT:
\end{verbatim}

and these are the associated examples with GPT-4 generated CoT explanations:

\begin{verbatim}
COMMENT: [SENT] Defense getting absolutely bullied by a 
dude that looks like he sells solar panels
IN-GROUP: Jets
OUT-GROUP: Bears
WIN PROBABILITY: 71.5%
TARGET: [SENT] [IN] getting absolutely bullied by [OUT] 
that looks like [OUT] sells solar panels .
REF_EXPRESSIONS: ['Defense', 'a dude', 'he']
EXPLANATION: The commenter is probably talking about the 
in-group, since 'Defense' is said without qualification, 
and the description of the offensive player is disparaging 
('he sells solar panels'). 'Defense' should be tagged [IN] 
since it refers to in-group, and 'a dude' and 'he' should 
be tagged [OUT] since it refers to an out-group offensive 
player.

COMMENT: [SENT] Hasn’t really been him . [SENT] Receivers 
have been missing a lot of easy catches.
IN-GROUP: Dolphins
OUT-GROUP: Chargers
WIN PROBABILITY: 49.21%
TARGET: [SENT] Hasn’t really been [IN] . [SENT] [IN] have 
been missing a lot of easy catches .
REF_EXPRESSIONS: ['him', 'Receivers']
EXPLANATION: The second sentence is complaining about the 
receivers missing a lot of catches, thus absolving another 
player of some blame, which is something fans would only do 
for the in-group team they support. Thus 'him' in first 
sentence, and 'Receivers' in second sentence should be 
tagged with [IN].

COMMENT: [SENT] Cards and rams are gonna be in the post-
season regardless, so I don't really care about them losing 
unless they play us.
IN-GROUP: 49ers
OUT-GROUP: Jaguars
WIN PROBABILITY: 99.71%
TARGET: [SENT] [OTHER] and [OTHER] are gonna be in the 
post-season regardless, so I don't really care about 
[OTHER] losing unless they play [IN].
REF_EXPRESSIONS: ['Cards', 'rams', 'them']
EXPLANATION: The game is between the 49ers and Jaguars, 
while the words 'Cards' and 'rams' refers to other teams 
in the NFL. Thus they should be tagged [OTHER] since they 
are neither in-group nor out-group, as should the word 
'them'. 'us' should be tagged [IN] since it refers to 
the in-group team the player supports.

COMMENT: [SENT] How are we this shit on defense
IN-GROUP: Steelers
OUT-GROUP: Eagles
WIN PROBABILITY: 4%
TARGET: [SENT] How are [IN] this shit on defense
REF_EXPRESSIONS: ['we']
EXPLANATION: 'we' here, and almost always, refers to 
the in-group since they don't like their team's defense, 
which is reflected in the low win probability. 'we' should 
therefore be tagged with [IN] since it refers to in-group.

COMMENT: [SENT] The chiefs got straight fucked with that 
Herbert INT getting called dead . [SENT] Suck it , KC !	
IN-GROUP: Chargers
OUT-GROUP: Chiefs
WIN PROBABILITY: 43.2%
TARGET: [SENT] [OUT] got straight fucked with that [IN] 
INT getting called dead . [SENT] Suck it , [OUT] !
REF_EXPRESSIONS: ['The chiefs', 'Herbert', 'KC']
EXPLANATION: This is a game between the Chiefs and the 
Chargers, and the commenter is a supporter of the Chiefs, 
so 'the chiefs' in the first sentence and 'KC' in the 
second sentence should be tagged [OUT]. Herbert is a player 
for the Chargers, and should be tagged with [IN] since 
he is a member of the in-group with respect to the 
commenter.

COMMENT: [SENT] Need points but 7 would be HUGE momentum
IN-GROUP: Bengals
OUT-GROUP: Chiefs
WIN PROBABILITY: 21.5%
TARGET: [IN] Need points but 7 would be HUGE momentum
REF_EXPRESSIONS: ['[SENT]']
EXPLANATION: The in-group team is losing currently as the 
win probability shows, so this comment is implicitly about 
the in-group needing points to gain momentum. Thus '[SENT]' 
should be tagged with '[IN]' since there is no explicit 
word/phrase that refers to the in-group, but the comment 
is referring to the in-group implicitly.

Some comments will have no explicit or implicit reference 
to the in-group, out-group, or other, or it could be 
extremely hard to disambiguate any references based on 
given information. In such cases, return TARGET as a copy 
of COMMENT, and justify this with the EXPLANATION "No 
explicit or implicit references to tag.", and return [] 
for REF_EXPRESSIONS. Here is an example:

COMMENT: [SENT] I thought so. [SENT] Wish I could say the 
same ;)
IN-GROUP: Jaguars
OUT-GROUP: Titans
WIN PROBABILITY: 41.5%
TARGET: [SENT] I thought so. [SENT] Wish I could say the 
same ;)
REF_EXPRESSIONS: []
EXPLANATION: No explicit or implicit references to tag.

Now tag only the relevant entities mentioned in the 
following comment as either in-group ([IN]), out-group 
([OUT]), or other ([OTHER]). Provide the tagged 
comment, REF_EXPRESSIONS and EXPLANATION accordingly after 
'TARGET: '.
\end{verbatim}

For the various model ablations (\texttt{fewshot}, \texttt{fewshot-instruct}, \texttt{fewshot-cot}), we used the above instructional prefix without the elements not included in that ablation --- so \texttt{fewshot} model would only have the examples, with no instructions, and no EXPLANATION.
